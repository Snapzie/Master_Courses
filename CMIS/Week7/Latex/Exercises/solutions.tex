\subsection{Matrix assembly versus update schemes}
When solving our partial differential equations for space and they are not time dependent, we can usually apply two methods to find solutions. The first one being to derive an update scheme which describes which other nodes in our mesh another node depends on. This can be combined into a stencil which can be moved over the entire mesh iteratively until no nodes in our mesh changes values. This requires us to make an initial guess about their values which, if far from the correct solution, can cause slow convergence.\\
Another way to find a solution is to note that a stencil form a linear system of equations which can be assembled into a matrix system of the form $\mathbf{A}\mathbf{x} = \mathbf{b}$ where we would be able to solve for $\mathbf{x}$.\\
The first approach is known as a \textit{update scheme} approach, and the second is known as a \textit{matrix assembly}. The advantage of using an update scheme approach is we do not need storage for a matrix which can end up being quite big as it need to hold values for all nodes in our mesh. However, it might take many iterations before an update scheme converges. At the same time, if we do matrix assembling we can apply different solvers to the matrix system like LU, QR or Cholesky factorization which might speed up process quite significantly.