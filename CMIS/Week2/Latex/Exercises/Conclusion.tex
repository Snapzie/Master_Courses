\section{Conclusion}
We end by concluding that advection without any boundary issues behaves differently than advection where we are affected by boundaries. This is due to the loss of information as our peaks move outside the boundaries, and due to how we interpolate back onto our grid, where if we interpolate back to the nearest grid cell might end up computing the value the cell in a slightly wrong location. We also conclude that using different boundary conditions when applying mean curvature flow can give vastly different end results. We find using $\frac{\partial \phi}{\partial x} = 0$ would give the most natural boundary and thus the best results, however, can be hard to implement. Thus using von Neumann boundary conditions is a good alternative as it does not make any abrupt change around the neighbourhood which a Dirichlet boundary condition would.