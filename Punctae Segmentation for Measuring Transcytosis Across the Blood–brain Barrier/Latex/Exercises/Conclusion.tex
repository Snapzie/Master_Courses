\section{Conclusion}
In conclusion we have seen how transfer learning helps improve the performance of models and conclude based on the validation losses in \autoref{imagenet} that the modified U-Net model performs best for our task. We have also seen how training on augmented data only makes this modified U-Net model perform worse. We conclude this is likely due to the original training set only containing a few images with some important features, and substituting in augmented images adds more of the same features instead of adding a more comprehensive set of possible features. We also conclude based on the results in \autoref{circlecount} that training on augmented data, although not visible in the losses, does improve performance on images with only a few dots, whereas training on the original training set increases model performance on image with a lot of dots. To end with, we conclude these models can not be used as an automatic tool to annotate two-photon microscopy images, but they can be used to give an initial count of puncta in the images although with some uncertainty.