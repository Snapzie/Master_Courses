\section{Introduction}
In this thesis project we will look into how machine learning can be used in the treatment of colitis. Colitis is a chronic digestive disease which is characterized by inflammation of the inner lining of the colon. Suffering from colitis causes pain and discomfort to the abdomen and often causes diarrhea with or without blood\cite{colitis}. Often four different treatments are prescribed, depending on the scope of inflammation in the colon. Local 5-ASA is a mild suppository if the inflammation does not stretch too far into the colon. Oral 5-ASA is a mild treatment and is prescribed if the inflammation is mild but stretches far into the colon. Oral steroid is a moderate treatment and is prescribed if the inflammation is severe, but not necessarily stretched far into the colon. Lastly IV steroids is a harsh treatment prescribed if the inflammation is severe and is stretched far into the colon.\\
In this thesis work, several deep neural networks were trained to detect inflammation in individual frames of endoscopy examinations in an attempt to draw segmentations of these videos. As the inflammation stretches into the colon without 'holes', it was attempted to predict this separation point, where we go from an inflamed colon to a healthy colon. Such a segmentation and separation point could then be used by a doctor to determine the correct treatment. However, the segmentations were also used to train other machine learning models to predict which treatment patients should be prescribed. One could imagine these results being used as a second opinion for a doctor.

The thesis work is divided into five sections. In the first, the theory used for this project is presented. In the second section the methods used throughout the thesis work is presented and goes into details about data pre-processing, how separation point predictions were made,  how treatment predictions were made and how the predictions were refined and then used for predictions again. Next the results of the thesis work will be presented before we go into a discussion about the results and end with a conclusion.