\section{Abstract}
In this thesis it was investigated how machine learning could be used in the treatment and diagnosis of colitis. Colitis is a chronic digestive disease and is characterised by inflammation in the colon. This characteristic was exploited and it was first investigated how well a 2D ResNet model could classify individual frames from an endoscopy examination. The best performing model achieved $65\%$ accuracy, although visual inspection showed the model was overfitting the training data and overly predicting inflammation. The resulting segmentation of the endoscopy videos were then used to train a 1D ResNet model predicting which of five treatments the patient had received. When trained only on predicted data the model achieved $40\%$ accuracy, and when adding the artificially constructed true segmentations to the training data, the model achieved $48\%$ accuracy. It was then attempted to train a 1D U-Net model to refine the segmentations and bring more structure to them. By visual inspection this was to some extent achieved although overly many frames were overturned to be healthy. Lastly, using these refined segmentation results, a new 1D ResNet model was trained to again predict the treatment received by patients. This time achieving $50\%$ accuracy when only trained on predicted data and $20\%$ accuracy when the true segmentations were added.