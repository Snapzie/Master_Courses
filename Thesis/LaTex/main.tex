% Opsætter KU Tex dokument
%%%%%%%%%%%%%%%%%%%%%%%%%%%%%%%%%%%%%%%%%%%%%%%%%%%%%%%%%%%%%%%%%%%%%%%%%%%%%%%%
\documentclass{article}                                                        %
\usepackage[a4paper, hmargin={2.8cm, 2.8cm}, vmargin={2.5cm, 2.5cm}]{geometry} %
\usepackage{eso-pic}  % \AddToShipoutPicture                                   %
\usepackage{graphicx} % \includegraphics                                       %
%\usepackage{subfig} - Can not be used with subcaption package (for subfigures)
\usepackage{setspace}                                                        %
%%%%%%%%%%%%%%%%%%%%%%%%%%%%%%%%%%%%%%%%%%%%%%%%%%%%%%%%%%%%%%%%%%%%%%%%%%%%%%%%

% Pakker til skrifttyper, tekst osv.
%%%%%%%%%%%%%%%%%%%%%%%%%%%%%%%%%%%%%%%%%%%%%%%%%%%%%%%%%%%%%%%%%%%%%%%%%%%%%%%%
    \usepackage[utf8]{inputenc}  % Implementere Unicode                        %
    \usepackage[T1]{fontenc}     % Unicode skrifttype fx. é skrives som 1 tegn %
    \usepackage[english]{babel}   % Dansk Ordbog                                %
    \usepackage{microtype}       % Forbedre linjeombrydningen                  %
    \usepackage{libertine}       % Skrifttype                                  %
%%%%%%%%%%%%%%%%%%%%%%%%%%%%%%%%%%%%%%%%%%%%%%%%%%%%%%%%%%%%%%%%%%%%%%%%%%%%%%%%

% Pakker til matematik og kode.
%%%%%%%%%%%%%%%%%%%%%%%%%%%%%%%%%%%%%%%%%%%%%%%%%%%%%%%%%%%%%%%%%%%%%%%%%%%%%%%%
    \usepackage{mathtools}       % Udvidelse til amsmath pakken                %
    \usepackage{amsthm}          % Pakke til bevisførelse                      %
    \usepackage{amssymb}         % Extra matematiske symboler                  %				
	\usepackage{wrapfig}													   %
	\usepackage{siunitx}	
	\usepackage{anyfontsize}
	\usepackage{ragged2e}
	\usepackage{algorithm2e}
	\usepackage[final]{pdfpages}
	\usepackage{listings}
	\usepackage{tikz}
	\usepackage{multirow}
	\usepackage{makecell}
	%\usepackage{fourier} 
	\usepackage{array}
	\usepackage{todonotes}
	\usepackage{pdflscape}
	\usetikzlibrary{arrows,shapes}
	\usepackage{titlesec}
	\usepackage{hyperref}
	\usepackage{url}
	\usepackage[nottoc,numbib]{tocbibind}
	\usepackage{semantic}
	\usepackage{subcaption}
	\usepackage{qtree}
	\usepackage{hhline}
	\usepackage{float}
    
	
	\definecolor{light-gray}{gray}{0.85}
	\lstset{
	    	numbers=left,
	    	breaklines=true,
	    	backgroundcolor=\color{light-gray},
	    	tabsize=2,
	    	basicstyle=\ttfamily,
	    	literate={\ \ }{{\ }}1
	}
	
	\urlstyle{same}
	\tikzstyle{vertex}=[circle,fill=white!25,minimum size=20pt,inner sep=0pt]
	\tikzstyle{edge} = [draw,thick,-]

	\definecolor{light-gray}{gray}{0.85}
	\definecolor{dkgreen}{RGB}{0.0,128.0,43.0}
	\lstdefinelanguage{FSharp}%
{morekeywords={new, match, with, rec, open, module, namespace, type, of, member, % 
and, for, while, true, false, in, do, begin, fun, function, return, yield, try, %end
mutable, if, then, else, cloud, async, static, use, abstract, interface, inherit, finally, int },
  otherkeywords={ let!, return!, do!, yield!, use!, var, from, select, where, order, by },
  keywordstyle=\color{bluekeywords},
  sensitive=true,
  basicstyle=\ttfamily,
	breaklines=true,
  xleftmargin=\parindent,
  aboveskip=\bigskipamount,
	tabsize=4,
  morecomment=[l][\color{dkgreen}]{///},
  morecomment=[l][\color{dkgreen}]{//},
  morecomment=[l][\color{dkgreen}]{(*},
  morecomment=[s][\color{dkgreen}]{},
  morestring=[b]",
  showstringspaces=false,
  literate={`}{\`}1,
  stringstyle=\color{redstrings},
}
	
	
												   %
%%%%%%%%%%%%%%%%%%%%%%%%%%%%%%%%%%%%%%%%%%%%%%%%%%%%%%%%%%%%%%%%%%%%%%%%%%%%%%%%

% Pakker til layout.
%%%%%%%%%%%%%%%%%%%%%%%%%%%%%%%%%%%%%%%%%%%%%%%%%%%%%%%%%%%%%%%%%%%%%%%%%%%%%%%%
    \usepackage{fancyhdr}            % Gør det muligt at bruge sidehoveder     %
    \usepackage{graphicx}            % Mulighed for bl.a. \includegraphics     %
    \usepackage{colortbl}            % Hvis man vil farvelægge sine tabeller   %
    \usepackage{array}               % Gør miljøerne array og tabular bedre    %
    \usepackage{parskip}             % Første paragraf i afsnit indrykkes ikke %
    \usepackage{titlesec}            % Tilpassing af afstand mellem sektioner  %
    \usepackage[lastpage,user]{zref} % Side x af y                             %
%%%%%%%%%%%%%%%%%%%%%%%%%%%%%%%%%%%%%%%%%%%%%%%%%%%%%%%%%%%%%%%%%%%%%%%%%%%%%%%%


% Implementerer en række makroer og de pakker der er importeret
%%%%%%%%%%%%%%%%%%%%%%%%%%%%%%%%%%%%%%%%%%%%%%%%%%%%%%%%%%%%%%%%%%%%%%%%%%%%%%%%
    \pagestyle{fancy}                        % Implementerer sidehoved         %
    \lhead{University of Copenhagen}                % Venstre sidehoved               %
    \rhead{Casper Bresdahl}                             % Højre sidehoved      %
    \cfoot{\thepage\ of \zpageref{LastPage}} % Side x af y                     %
    \newtheorem*{prp}{Propostion}            % Skaber nyt theorem  
    \renewcommand{\baselinestretch}{1.25}       %
%%%%%%%%%%%%%%%%%%%%%%%%%%%%%%%%%%%%%%%%%%%%%%%%%%%%%%%%%%%%%%%%%%%%%%%%%%%%%%%%

% Mindsker afstanden mellem sektioner
%%%%%%%%%%%%%%%%%%%%%%%%%%%%%%%%%%%%%%%%%%%%%%%%%%%%%%%%%%%%%%%%%%%%%%%%%%%%%%%%%%
\titlespacing\section{0pt}{12pt plus 4pt minus 2pt}{0pt plus 1pt minus 3pt}      %
\titlespacing\subsection{0pt}{12pt plus 4pt minus 2pt}{0pt plus 1pt minus 3pt}   %
\titlespacing\subsubsection{0pt}{12pt plus 4pt minus 2pt}{0pt plus 1pt minus 3pt}%
%%%%%%%%%%%%%%%%%%%%%%%%%%%%%%%%%%%%%%%%%%%%%%%%%%%%%%%%%%%%%%%%%%%%%%%%%%%%%%%%%%

%Ændrer størelsen på sections
%%%%%%%%%%%%%%%%%%%%%%%%%%%%%%%%%%%%%%%%%%%%%%%%%%%%%%%%%%%%%%%%%%%%%%%%%%%%%%%%%%
\titleformat{\section}
{\normalfont\fontsize{14}{16}\bfseries}{\thesection}{1em}{}
\titleformat{\subsection}
{\normalfont\fontsize{12}{14}\bfseries}{\thesubsection}{1em}{}
\titleformat{\subsubsection}
{\normalfont\fontsize{11}{13}\bfseries}{\thesubsubsection}{1em}{}
%%%%%%%%%%%%%%%%%%%%%%%%%%%%%%%%%%%%%%%%%%%%%%%%%%%%%%%%%%%%%%%%%%%%%%%%%%%%%%%%%%

%%%%%%%%%%%%
% Document %
%%%%%%%%%%%%

\begin{document}

\begin{titlepage}

\newcommand{\HRule}{\rule{\linewidth}{0.5mm}} % Defines a new command for the horizontal lines, change thickness here

\begin{center}
 % Center everything on the page
 
%----------------------------------------------------------------------------------------
%	HEADING SECTIONS
%----------------------------------------------------------------------------------------

\textsc{\LARGE University of Copenhagen}\\[1.5cm] % Name of your university/college
\textsc{\Large Computer Science}\\[0.5cm] % Major heading such as course name
\textsc{\large Master's Thesis}\\[0.5cm] % Minor heading such as course title

%----------------------------------------------------------------------------------------
%	TITLE SECTION
%----------------------------------------------------------------------------------------

\HRule \\[0.4cm]
{ \huge \bfseries Colitis Treatment Predictions Using Machine Learning and Endoscopy Videos}\\[0.4cm] % Title of your document
\HRule \\[1.5cm]
 
%----------------------------------------------------------------------------------------
%	AUTHOR SECTION
%----------------------------------------------------------------------------------------

\begin{minipage}{0.4\textwidth}
\begin{flushleft} \large
\emph{Author:}\\ 
% Your name
Casper \textsc{Bresdahl} \\
\end{flushleft}
\end{minipage}
~
\begin{minipage}{0.4\textwidth}
\begin{flushright} \large
\emph{Supervisor:} \\
Bulat \textsc{Ibragimov}
\end{flushright}
\end{minipage}\\[2cm]

% If you don't want a supervisor, uncomment the two lines below and remove the section above
%\Large \emph{Forfattere:}\\
%Axel \textsc{Christof}\\% Your name
%Casper \textsc{Bresdahl}\\
%Emilie \textsc{Bentsen}\\[1cm] 

%----------------------------------------------------------------------------------------
%	DATE SECTION
%----------------------------------------------------------------------------------------

{\large \today}\\[2cm] % Date, change the \today to a set date if you want to be precise

%----------------------------------------------------------------------------------------
%	LOGO SECTION
%----------------------------------------------------------------------------------------

\includegraphics{logo.png}\\[1cm] % Include a department/university logo - this will require the graphicx package
 
%----------------------------------------------------------------------------------------

\vfill % Fill the rest of the page with whitespace

%Disse linjer skaber forside, evt indholdsfortegnelse, og sætter sidetal
%%%%%%%%%%%%%%%%%%%%%%%%%%%%%%%%%%%%%%%%%%%%%%%%%%%%%%%%%%%%%%%%%%%%%%%%%%%%%%%%
									                                           %
    \thispagestyle{empty}   % Fjerner sidetal forside                          %
        % Slå disse til hvis der ønskes indholdsfortegnelse                    %
        %%%%%%%%%%%%%%%%%%%%%%%%%%%%%%%%%%%%%%%%%%%%%%%%%%%%%%%%%%%%%%%%%%%%%%%% 
            \newpage                % Side til indholdsfortegnelse            %
            \thispagestyle{empty}   % Fjerner sidetal fra indholdsfortegnelse %
            \tableofcontents        % Skaber indholdsfortegnelse              %
    \end{center}
    %\section*{Forord}
    	
		
        %%%%%%%%%%%%%%%%%%%%%%%%%%%%%%%%%%%%%%%%%%%%%%%%%%%%%%%%%%%%%%%%%%%%%%%%
    \newpage                % Første rigtige side
    \setcounter{page}{1}    % Sætter rigtigt sidetal på første side
%%%%%%%%%%%%%%%%%%%%%%%%%%%%%%%%%%%%%%%%%%%%%%%%%%%%%%%%%%%%%%%%%%%%%%%%%%%%%

\end{titlepage}
{\fontsize{10}{14}\selectfont
\section{Abstract}
In this thesis it was investigated how machine learning could be used in the treatment and diagnosis of colitis. Colitis is a chronic digestive disease and is characterised by inflammation in the colon. This characteristic was exploited and it was first investigated how well a 2D ResNet model could classify individual frames from an endoscopy examination. The best performing model achieved $65\%$ accuracy, although visual inspection showed the model was overfitting the training data and overly predicting inflammation. The resulting segmentation of the endoscopy videos were then used to train a 1D ResNet model predicting which of five treatments the patient had received. When trained only on predicted data the model achieved $40\%$ accuracy, and when adding the artificially constructed true segmentations to the training data, the model achieved $48\%$ accuracy. It was then attempted to train a 1D U-Net model to refine the segmentations and bring more structure to them. By visual inspection this was to some extent achieved although overly many frames were overturned to be healthy. Lastly, using these refined segmentation results, a new 1D ResNet model was trained to again predict the treatment received by patients. This time achieving $50\%$ accuracy when only trained on predicted data and $20\%$ accuracy when the true segmentations were added.
\section{Motivation}
The blood-brain barrier is an interface separating the brain form the circulatory system \cite{swhati}. The very restrictive nature of the blood-brain barrier, only allowing certain small molecules to pass, makes it a big obstacle in central nervous system drug delivery \cite{swhati}. Defects in the blood-brain barrier have also been reported in Alzheimer's and Parkinson's disease patients, connecting the two diseases to the blood-brain barrier \cite{zhao}. For these reasons research in the blood-brain barrier and its permeability is important. When studying the permeability of the blood-brain barrier, tracer molecules are injected into the brain of mice which then accumulate around the blood-brain barrier interface. With two-photon microscopy we can see and measure the concentration of these tracer molecules. However, quantifying these accumulations is very subjective, non-trivial and time consuming, even for experts, and for these reasons this project aims to explore the possibilities of training and validating a deep learning model for automatic segmentation of accumulation of these tracer molecules.  
\section{Theory}
\subsection{Conservation properties of FVM}
We will begin by going into detail about the conservation properties of FVM. When we have a large global volume and split it into several smaller local volumes we introduce new common surfaces between the local volumes. This have an effect when we apply the Gauss-Divergence theorem to go from volume integrals to surface integrals as we now integrate over the surfaces. To illustrate this we can define some vector function $\mathbf{f}(\mathbf{u})$ and integrate the divergence of this function over some fixed volume \textit{V} giving us:
\begin{equation*}
	\int_V \nabla\cdot\mathbf{f}(\mathbf{u})dV
\end{equation*}
Converting this to a surface integral we apply Gauss-Divergence theorem giving us:
\begin{equation*}
	\int_S \mathbf{f}(\mathbf{u})\cdot\mathbf{n}dS
\end{equation*}
Where \textbf{n} is the outwards unit normal. However, as discussed and illustrated in \autoref{conservation} by splitting the global volume we have introduced a new surface. Looking at  \autoref{conservation} we have $V_a$ and $V_b$ share the common surface $S_c$, the boundary of $V_a$ is $S_a \cup S_c$ and the boundary of $V_b$ is $S_b \cup S_c$. We also note $V = V_a \cup V_b$ and $S = S_a \cup S_b$. We can thus integrate over the local volumes and get:
\begin{align*}
	\int_{V_a\cup V_b}\nabla\cdot\mathbf{f}(\mathbf{u})dV &= \int_{S_a} \mathbf{f}(\mathbf{u})\cdot\mathbf{n_a}dS + \int_{S_c} \mathbf{f}(\mathbf{u})\cdot\mathbf{n_c}dS\\ 
	&+\int_{S_b} \mathbf{f}(\mathbf{u})\cdot\mathbf{n_b}dS + \int_{S_c} \mathbf{f}(\mathbf{u})\cdot(-\mathbf{n_c})dS
\end{align*}
We here see the integrals over the common surface cancels out giving us:
\begin{equation*}
	\int_{V_a\cup V_b}\nabla\cdot\mathbf{f}(\mathbf{u})dV = \int_{S_a\cup S_b} \mathbf{f}(\mathbf{u})\cdot\mathbf{n}dS
\end{equation*}
Which is equivalent to:
\begin{equation*}
	\int_V\nabla\cdot\mathbf{f}(\mathbf{u})dV = \int_S \mathbf{f}(\mathbf{u})\cdot\mathbf{n}dS
\end{equation*}
This means conservation is guaranteed on both a local and a global scale.

\begin{figure}
	\centering
	\includegraphics[width=0.7\linewidth]{Materials/Conservation}
	\caption{Illustration of a global volume getting split introducing a new common surface.}
	\label{conservation}
\end{figure}

\subsection{Control volumes}
An integral part of FVM is the control volumes. Control volumes are introduced as part of our discretization to simplify our problem such that we look at a local subproblem which is easier to solve. Control volumes can be of any shape as long as we have the geometric information of them, e.g. cell centers, surface centers, neighbouring cells, surface connectivity etc. For this week's assignment we will construct a regular triangle mesh. In this mesh we will define the control volumes 'around' vertices of the mesh. That is, in the interior of the mesh we will construct octagons and squares going between the centers of our triangle mesh. At the boundaries of our mesh we will create polygons. This can be seen in \autoref{controlvolume}. Although there are no formal requirements on the shape of the control volumes we need the edges of the control volumes to be perpendicular to the grid lines in our mesh, otherwise the outwards unit normal of the control volume surface will point in a different direction than the line segments connecting our vertices and we will introduce errors in our later approximations. Because of this we need to tailor the shape of our control volumes such that they match with our choice of mesh.

\subsection{FVM applied to Magnetostatic problem}
We can now apply FVM to our problem in this week's notebook, namely the Magnetostatic problem which has the PDE $\nabla \cdot \nabla \phi(\mathbf{x}) = \nabla \cdot \mathbf{M}(\mathbf{x})$. The first step of FVM is to decide on a mesh and control volume layout. As we are dealing with a 2D problem in the notebook we will use a regular triangle mesh. For control volumes we will use the centers in the triangles to draw octagons and squares between interior nodes, and simple lines to the boundary. This is illustrated in \autoref{controlvolume} where black lines is our mesh and red lines constitute our control volumes. The next step is to convert our PDE to a volume integral and then to a surface integral using the \textit{Gauss-Divergence theorem}.
\begin{align*}
	\nabla \cdot \nabla \phi(\mathbf{x}) &= \nabla \cdot \mathbf{M}(\mathbf{x})\\
	\int_V\nabla \cdot \nabla \phi(\mathbf{x})dV &= \int_V \nabla \cdot \mathbf{M}(\mathbf{x})dV\\
	\int_S\nabla \phi(\mathbf{x})\cdot \mathbf{n}dS &= \int_S \mathbf{M}(\mathbf{x})\cdot \mathbf{n}dS
\end{align*}
Where $\mathbf{n}$ is the outward unit normal. These surface integrals means we need to integrate over the edges of our control volumes. We can now exploit that the edges are piecewise continuous, making our integrals \textit{piecewise continuous integrals}. 
\begin{equation*}
	\sum_e\int_{S_e}\nabla \phi(\mathbf{x})\cdot \mathbf{n_e}dS = \sum_e\int_{S_e} \mathbf{M}(\mathbf{x})\cdot \mathbf{n_e}dS
\end{equation*}
Where $\mathbf{n_e}$ denotes the outward unit normal for the edges in the control volume. We can now use the midpoint approximation rule to remove the integral. This can be done because the outward unit normal is constant along each $S_e$ part. 
\begin{equation*}
		\sum_e[\nabla \phi(\mathbf{x})\cdot \mathbf{n_e}]_cl_e = \sum_e [\mathbf{M}(\mathbf{x})\cdot \mathbf{n_e}]_cl_e
\end{equation*}
Here $l_e$ denote the length of the control volume edge we are looking at. To approximate $[\nabla \phi(\mathbf{x})\cdot \mathbf{n_e}]_c$ we realise we can rewrite it as a directional derivative. We note because of the way we have defined our control volumes we have $n_e$ points in the same direction as the line segment between two vertices and we can thus make the approximation:
\begin{equation*}
	\nabla\phi(\mathbf{x})\cdot \mathbf{n_e} = \frac{\partial\phi(\mathbf{x})}{\partial \mathbf{n_e}} \approx \frac{\phi(\mathbf{x})_j - \phi(\mathbf{x})_i}{l_{ij}}
\end{equation*}
Where $\phi(\mathbf{x})_i$ is the $\phi$ value at a vertex \textit{i} in our mesh, $\phi(\mathbf{x})_j$ is the $\phi$ value at a vertex \textit{j} in our mesh and $l_{ij}$ is the length of the line segment connecting the two vertices. Substituting this in and cleaning up we end with the discretization:
\begin{align*}
	\sum_e\frac{(\phi(\mathbf{x})_j - \phi(\mathbf{x})_i)}{l_{ij}}l_e &= \sum_e [\mathbf{M}(\mathbf{x})\cdot \mathbf{n_e}]_cl_e\\
	\sum_e\frac{l_e}{l_{ij}}(\phi(\mathbf{x})_j - \phi(\mathbf{x})_i) &= \sum_e [\mathbf{M}(\mathbf{x})\cdot \mathbf{n_e}]_cl_e
\end{align*}
This approximation does however introduce small error along the boundary. This is illustrated in \autoref{boundary} where we end up estimating the $\phi$ value only halfway to the boundary (red dot) instead of at the boundary (green dot) because the control volume edge only goes to the boundary. We could instead use a more complicated approximation where we use shape functions like in FEM to interpolate the value to the boundary. This would mean we would have $\nabla \phi(\mathbf{x}) = \sum_\alpha \nabla N_{\alpha}\hat{\phi}_{\alpha} = \mathbf{N}\hat{\phi}$ where $N_{\alpha}$ is a shape function and $\hat{\phi}$ is a set of discrete $\phi$ values we would then solve for. 

\begin{figure}
	\centering
	\includegraphics[width=0.5\linewidth]{Materials/ControlVolume}
	\caption{Our mesh drawn in black and our control volumes drawn in red.}
	\label{controlvolume}
\end{figure}

\begin{figure}
	\centering
	\includegraphics[width=0.5\linewidth]{Materials/boundary}
	\caption{Small error introduced by not using shape functions for interpolation. Blue lines illustrate edges in control volumes and the black lines illustrate a single boundary triangle in our mesh.}
	\label{boundary}
\end{figure}

\subsection{Handling of unit circle}
In this week's assignment when we look at the right hand side integral, $\int_V \nabla \cdot \mathbf{M}(\mathbf{x})dV$, we require the integrand, $\nabla \cdot \mathbf{M}(\mathbf{x})$, to be a continuous real valued function. When we consider a control volume completely outside the unit circle $\mathbf{M}(\mathbf{x})$ is per definition $[0,0]^T$ and thus continuous. When we consider a control volume completely inside the unit circle $\mathbf{M}(\mathbf{x})$ is per definition $[0,-1]^T$ and thus also continuous. However, when we have a control volume which is partially inside and partially outside then $\mathbf{M}(\mathbf{x})$ suddenly jumps from one value to another and is thus discontinuous. We can handle this by splitting the integral into two parts, one over the part of the control volume which is outside the unit circle, and one which is inside and then sum the two parts. That is we would have:
\begin{equation*}
	\int_V \nabla \cdot \mathbf{M}(\mathbf{x})dV = \int_{V_{inside}} \nabla \cdot \mathbf{M}(\mathbf{x})dV + \int_{V_{outside}} \nabla \cdot \mathbf{M}(\mathbf{x})dV
\end{equation*}
This split can either be done on the fly when we realise our control volume is getting split by the unit circle, or we could preemptively design our control volumes such that their boundaries conforms with the unit circle.
\subsection{Difference between Finite Methods}
FVM resembles some of the same concepts as FEM as in both methods we start with a big system which we then divide into smaller subsystems which we then can solve more easily. However, where FEM uses trial functions and integration by parts to convert from strong form to weak form, FVM takes a more direct approach and uses Gauss-Divergence theorem to get surface integrals. FEM also relies on the approximation from using shape functions to interpolate positions inside the domain, where FVM can make use of shape functions, but can also as seen earlier, make use of the outward unit normal from the control volumes. Comparing to FDM, FVM and FEM are much more easily applied to unstructured meshes as it is not always obvious which nodes to use for FDM in an unstructured mesh. However, FDM can quite easily be formally verified though analysis of the Taylor remainder terms, which is not as obviously done for FVM and FEM.
\section{Methods}
The goal of this project is to train a model which as input takes the two-photon microscopy images of mice brains and outputs the locations of vescicles puncta. To train the model an annotated dataset is needed, where each puncta has been marked by hand by an expert in the field. The CNN is then trained by supervised learning to predict the location of the puncta. This section describes the methods used to achieve this.

\subsection{Preprocessing}
As the two-photon microscopy images comes in many different resolution depending on equipment used, preprocessing is needed to standardize the images. First all images have been scaled to the same resolution. Then, by finding the largest image, all other images has been padded with zeros until they achieve the same size. The images have then been resized to a size of 1024 by 1088. To ensure the masks still correspond to the images, the same scaling, padding and resizing have been applied to the masks. The images only have one colour channel with intensities between 0 and 4095. To make this match the input layer of the modified U-Net model, the intensities are normalized to be between 0 and 1 by first clamping all pixel intensities to be between 0 and 4095 and then dividing each pixel in the images with 4095. As the masks are binary to mark the location of puncta, this last preprocessing step has not been applied to the masks. The, in total, 722 images are now split into a training set of 506 images, a validation set of 104 images and a test set of 112 images. After the initial preprocessing the images have been resized to 512 by 544, that is the images have been halved. A bicubic interpolation has been used for the images and a nearest neighbour interpolation has been used for the masks in all datasets. Regardless of image sizes, in order to make the images comply with the modified U-Net model, three colour channels are needed. As our images only have one colour channel, this channel has simply been replicated twice.

\subsection{Models} \label{models}
The first model developed was a modified U-Net model where the encoder part has been replaced by a ResNet-18 model, transposed convolutions have been used for the upsampling layers to learn how the images are upsampled optimally, a sigmoid function have been used as activation function for the final output layer as the masks are binary, Adam have been used for optimization with an initial learning rate of 0.0001, soft DICE have been used for loss function and its weights have been initialized with the ImageNet weights. Very similarly, the second model uses the same configuration except it is not initialized with ImageNet weights. The last two models are LinkNet models using the same configurations as the previous two models.

% To determine whether pre-trained ImageNet weights should be used, an experiment was conducted early on in the project, where two models were trained using unmodified training data. The results of the training can be seen in \autoref{imagenet} where we see the model whose weights is initialized with the ImageNet weights performs better on the validation set. For this reason, the rest of the models trained are trained with the ImageNet weights.

%\begin{figure}
%	\centering
%	\begin{subfigure}[b]{0.48\linewidth}
%		\centering
%		\includegraphics[width=\linewidth]{Materials/Method/UnetImagenet}
%		\caption{Model weights initialized with ImageNet weights.}
%	\end{subfigure}
%	\begin{subfigure}[b]{0.48\linewidth}
%		\centering
%		\includegraphics[width=\linewidth]{Materials/Method/UnetNoImagenet}
%		\caption{Model weights not initialized with ImageNet weights.}
%	\end{subfigure}
%	\caption{Both models are trained on unmodified training data. The blue graphs indicates training loss measured with DICE, and the orange graphs indicates validation loss also measured with DICE.}
%	\label{imagenet}
%\end{figure}

\subsection{Testing}
To measure the models unbiased performance on unknown data, the test set is used. As all predictions and true masks have image sizes 524 x 544 they are resized back to original sizes using the same interpolation rules. The 'immediate' output of the models are confidence maps, and to make these binary, we threshold them. This is done by thresholding all predicted masks from the training set and then computing the average DICE. The threshold that achieves the highest average DICE is then chosen, and all predicted masks from the test set are then thresholded by this value, and its unbiased performance can then be measured.\\
There exists different ways of measuring DICE on the test set. The test set can be split into three separate volumes, one for the single series images, one for the five series images and one for the six series images and the average DICE can then be measured across each. The three volumes can also further be split into 'time intervals' where the DICE is measured across all the images coming first in the series, second in the series and so on.

\subsection{Augmentations}
The first augmentation used is large random rotations. Here each of the original images have been rotated between $-45$ and $45$ degrees. Another dataset of small random rotations between $-30$ and $30$ degrees have also been created. The next augmentations flips each image horizontally and vertically giving us two more datasets. Two datasets have also been created for small and large random shearing in both the \textit{x} and \textit{y} direction. A \textit{mix} dataset has also been created where the original images were first sheared then rotated and finally flipped horizontally. A dataset was also created where the original images were 'bend' around the middle to form an arc shape. The last dataset created used the python package \textit{imgaug} to randomly apply one or more of the following augmentations to each image: horizontal flipping, vertical flipping, $80-120\%$ scaling, small translations, $-45$ to $45$ degrees rotations, $-16$ to $16$ degrees shearing, randomly drop $1-10\%$ of the pixels in the image, locally move some pixels around and lastly, perform local distortions. 

\subsection{Construction of augmented training sets}\label{construction}
With the augmented datasets, several training sets were constructed and used to train models. Each training set consists of 506 training images, and is constructed by randomly sample from the involved datasets without replacement, ensuring an even amount of images are taken from each dataset. An \textit{original} training set is also created consisting of the un-augmented images.
\section{Results}
\subsection{Initial results}
\begin{figure}
	\centering
	\begin{subfigure}[b]{0.24\linewidth}
		\centering
		\includegraphics[width=\linewidth]{Materials/Results/Initial/UnetImagenet}
		\caption{U-Net model whose weights are initialized with ImageNet weights.}
	\end{subfigure}
	\hfill
	\begin{subfigure}[b]{0.24\linewidth}
		\centering
		\includegraphics[width=\linewidth]{Materials/Results/Initial/UnetNoImagenet}
		\caption{U-Net model whose weights are not initialized with ImageNet weights.}
	\end{subfigure}
	\hfill
	\begin{subfigure}[b]{0.24\linewidth}
		\centering
		\includegraphics[width=\linewidth]{Materials/Results/Initial/LinknetImagenet}
		\caption{LinkNet model whose weights are initialized with ImageNet weights.}
	\end{subfigure}
	\hfill
	\begin{subfigure}[b]{0.24\linewidth}
		\centering
		\includegraphics[width=\linewidth]{Materials/Results/Initial/LinknetNoImagenet}
		\caption{LinkNet model whose weights are not initialized with ImageNet weights.}
	\end{subfigure}
	\caption{All models are trained on the original and halved training data. The blue graphs indicates training loss measured with DICE, and the orange graphs indicates validation loss also measured with DICE.}
	\label{imagenet}
\end{figure}
Early in the project the models described in \autoref{models} were trained on the original training set. In \autoref{imagenet} we see the training and validation loss measured in DICE for each of the models. We see that all models could probably still achieve lower training loss, but after 150 epochs both U-Net models achieve lower training loss than the LinkNet model using Imagenet weights, and a lot lower training loss than the LinkNet model not using ImageNet weights. Looking at the validation loss, the two models using ImageNet weights shows the lowest loss with the U-Net model using ImageNet weights having slightly lower loss than the LinkNet model using ImageNet weights.

\begin{figure}[H]
	\centering
	\begin{subfigure}[b]{0.19\linewidth}
		\centering
		\includegraphics[width=\linewidth]{Materials/Results/Augmentation/OH}
		\caption{Original and horizontal data model.\\}
	\end{subfigure}
	\hfill
	\begin{subfigure}[b]{0.19\linewidth}
		\centering
		\includegraphics[width=\linewidth]{Materials/Results/Augmentation/OR}
		\caption{Original and large rotations model.\\}
	\end{subfigure}
	\hfill
	\begin{subfigure}[b]{0.19\linewidth}
		\centering
		\includegraphics[width=\linewidth]{Materials/Results/Augmentation/OS}
		\caption{Original and large shearing model.\\}
	\end{subfigure}
	\hfill
	\begin{subfigure}[b]{0.19\linewidth}
		\centering
		\includegraphics[width=\linewidth]{Materials/Results/Augmentation/ORS}
		\caption{Original, large rotation and large shearing model.}
	\end{subfigure}
	\hfill
	\begin{subfigure}[b]{0.19\linewidth}
		\centering
		\includegraphics[width=\linewidth]{Materials/Results/Augmentation/ORHV}
		\caption{Original, large rotation, horizontal and vertical model.}
	\end{subfigure}
	\\
	\begin{subfigure}[b]{0.19\linewidth}
		\centering
		\includegraphics[width=\linewidth]{Materials/Results/Augmentation/ORHVS}
		\caption{Original, large rotation, large shearing, horizontal and vertical model.}
	\end{subfigure}
	\hfill
	\begin{subfigure}[b]{0.19\linewidth}
		\centering
		\includegraphics[width=\linewidth]{Materials/Results/Augmentation/Mix}
		\caption{Mix model.\newline\newline\newline}
	\end{subfigure}
	\hfill
	\begin{subfigure}[b]{0.19\linewidth}
		\centering
		\includegraphics[width=\linewidth]{Materials/Results/Augmentation/Arc}
		\caption{Original and arc model.\newline\newline}
	\end{subfigure}
	\hfill
	\begin{subfigure}[b]{0.19\linewidth}
		\centering
		\includegraphics[width=\linewidth]{Materials/Results/Augmentation/O30R30S}
		\caption{Original, small rotations and small shearing model.\newline}
	\end{subfigure}
	\hfill
	\begin{subfigure}[b]{0.19\linewidth}
		\centering
		\includegraphics[width=\linewidth]{Materials/Results/Augmentation/Original}
		\caption{Original model.\newline\newline\newline}
	\end{subfigure}
	\caption{Results of training the modified U-Net model initialized with the ImageNet weights on different augmented datasets. The blue graphs indicates training loss measured with DICE, and the orange graphs indicates validation loss also measured with DICE.}
	\label{augres}
\end{figure}

\subsection{Augmentation results}
In \autoref{augres} are the results of training the modified U-Net model initialized with the ImageNet weights on different augmented training sets. We note the model achieving the lowest validation loss is the model trained on the original training set, achieving an average DICE around 0.5. The model trained on augmented data which achieves the lowest validation loss is the model trained on the original, the small rotations and the small shearing data sets, achieving an average DICE around 0.46. Although this model has been trained on 150 epochs it seems to give the same results after 100 epochs. The model trained on the original and horizontal data along with the model trained on the original and large shearing data also seems to come close to an average DICE of 0.45. Examples of prediction results can be seen in the appendix.\\
An experiment was conducted to see if it would make a difference in model performance to concatenate the datasets together rather than the fixed 506 images. In \autoref{concat} we see two models trained on the horizontal, small rotation and small shearing data sets, but one have been trained on 506 images whereas the other have had the three datasets concatenated and then shuffled. The most significant change is how fast the validation loss converges in the model trained on the concatenated dataset, although the validation loss ends being the same. We also note the training loss ends a lot higher in model trained on the concatenated data sets.

\begin{figure}
	\centering
	\begin{subfigure}[b]{0.35\linewidth}
		\centering
		\includegraphics[width=\linewidth]{Materials/Results/Augmentation/H30R30S}
		\caption{Modified U-Net model trained on the horizontal, small rotations and small shearing datasets.\newline}
	\end{subfigure}
	\qquad
	\begin{subfigure}[b]{0.35\linewidth}
		\centering
		\includegraphics[width=\linewidth]{Materials/Results/Augmentation/H30R30SC}
		\caption{Modified U-Net model trained on the horizontal, small rotations and small shearing datasets concatenated and shuffled.}
	\end{subfigure}
	\caption{Experiment showing concatenating the data sets rather than picking 506 images to train on yields same validation results.}
	\label{concat}
\end{figure}

\subsection{Test results}
\begin{wraptable}{l}{0.6\linewidth}
	\centering
	\resizebox{0.55\columnwidth}{!}{%
	\begin{tabular}{|c|c|c|c|c|c|c|c|}
		\hline
		\multicolumn{2}{|c|}{ } & $t_1$ & $t_2$ & $t_3$ & $t_4$ & $t_5$ & $t_6$ \\
		\hline
		\multirow{6}{*}{Fives data}
		&\cellcolor{light-gray}Zero augmented &\cellcolor{light-gray} 0.0 &\cellcolor{light-gray} 0.0 &\cellcolor{light-gray}   0.055 &\cellcolor{light-gray} 0.038 &\cellcolor{light-gray} 0.138 &\cellcolor{light-gray} \\ \hhline{~|-|-|-|-|-|-|-|}
		&\cellcolor{light-gray}Non-zero augmented &\cellcolor{light-gray} 0.0 &\cellcolor{light-gray} 0.0 &\cellcolor{light-gray}   0.387 &\cellcolor{light-gray} 0.132 &\cellcolor{light-gray} 0.240 &\cellcolor{light-gray} \\\hhline{~|-|-|-|-|-|-|-|}
		&\cellcolor{light-gray} Relative change & \cellcolor{light-gray} 0.0 &\cellcolor{light-gray} 0.0 &\cellcolor{light-gray} 0.332 &\cellcolor{light-gray} 0.094 &\cellcolor{light-gray} 0.102 &\cellcolor{light-gray} \\\cline{2-8}
		&Zero original			   & 0.0 & 0.0 &   0.105 & 0.082 & 0.123 & \\\cline{2-8}
		&Non-zero original	    & 0.0 & 0.0 &   0.366 & 0.191 & 0.216 & \\\cline{2-8}
		& Relative change &  0.0 & 0.0 & 0.261 & 0.109 & 0.093 & \\\hline
		\multirow{6}{*}{Sixes data}
		&\cellcolor{light-gray}Zero augmented 		  &\cellcolor{light-gray} 0.0  &\cellcolor{light-gray} 0.0 \cellcolor{light-gray}&\cellcolor{light-gray}   0.067 &\cellcolor{light-gray} 0.217 &\cellcolor{light-gray} 0.318 &\cellcolor{light-gray} 0.427 \\\hhline{~|-|-|-|-|-|-|-|}
		&\cellcolor{light-gray}Non-zero augmented &\cellcolor{light-gray} 0.0 &\cellcolor{light-gray} 0.0	 &\cellcolor{light-gray} 0.667 &\cellcolor{light-gray} 0.309 &\cellcolor{light-gray} 0.398 &\cellcolor{light-gray} 0.427 \\\hhline{~|-|-|-|-|-|-|-|}
		& \cellcolor{light-gray}Relative change &\cellcolor{light-gray} 0.0 &\cellcolor{light-gray} 0.0 &\cellcolor{light-gray} 0.600 &\cellcolor{light-gray} 0.092 &\cellcolor{light-gray} 0.080 &\cellcolor{light-gray} 0.0 \\\cline{2-8}
		&Zero original			   & 0.0 & 0.059& 0.147 & 0.239 & 0.289 & 0.439 \\\cline{2-8}
		&Non-zero original	    & 0.0 & 0.198& 0.367 & 0.299 & 0.362 & 0.439 \\\cline{2-8}
		&Relative change & 0.0 & 0.139 & 0.220 & 0.060 & 0.073 & 0.0 \\\hline
	\end{tabular}%
	}
	\caption{Average DICE of each 'time interval' of the five series and six series test data.}
	\label{tableDice}
\end{wraptable}

In \autoref{onesfivessixes} the test set has been split into three separate volumes. Each of the results have been reported with and without having zero results in the average. The model trained on augmented data seems to score slightly higher or the same averages on all volumes when we do not count in zero results whereas when we do count zero results the model trained on the original training set seems to score higher. The overall results seem to be quite similar. Test results for the rest of the models can be seen in the appendix. In \autoref{tableDice} the volumes have further been split into 'time intervals'. Here we see when removing zero results the relative increase in DICE is greatest for the model trained on augmented data, and the relative increase diminishes as we move 'up in time'. In \autoref{circlecount} the performance have been measured with the circle count metric. We note for both models the regression line is quite close to have slope 1. We also note for the model trained on the original training set, that in images where the true mask count is low the predicted number of circles is high, whereas when the true mask count is high, the predicted number of circles is close to be the true number of circles. For the model trained on the augmented data we see the opposite trend.

\begin{figure}
	\centering
	\begin{subfigure}[b]{\linewidth}
		\centering
		\includegraphics[width=\linewidth]{Materials/Results/Test/Originaltest}
		\caption{Results for the model trained on the original training set.}
	\end{subfigure}
	\\
	\begin{subfigure}[b]{\linewidth}
		\centering
		\includegraphics[width=\linewidth]{Materials/Results/Test/Augtest}
		\caption{Results for the model trained on the original, small rotations and small shearing data sets.}
	\end{subfigure}
	\caption{Average DICE on test set split into three separate volumes consisting of the single series images, five series images and six series images.}
	\label{onesfivessixes}
\end{figure}

%\begin{table}
%	\centering
%	\resizebox{0.65\columnwidth}{!}{%
%	\begin{tabular}{|c|c|c|c|c|c|c|c|}
%		\hline
%		\multicolumn{2}{|c|}{ } & $t_1$ & $t_2$ & $t_3$ & $t_4$ & $t_5$ & $t_6$ \\
%		\hline
%		\multirow{6}{*}{Fives data}
%			&\cellcolor{light-gray}Zero augmented &\cellcolor{light-gray} 0.0 &\cellcolor{light-gray} 0.0 &\cellcolor{light-gray}   0.055 &\cellcolor{light-gray} 0.038 &\cellcolor{light-gray} 0.138 &\cellcolor{light-gray} \\ \hhline{~|-|-|-|-|-|-|-|}
%			&\cellcolor{light-gray}Non-zero augmented &\cellcolor{light-gray} 0.0 &\cellcolor{light-gray} 0.0 &\cellcolor{light-gray}   0.387 &\cellcolor{light-gray} 0.132 &\cellcolor{light-gray} 0.240 &\cellcolor{light-gray} \\\hhline{~|-|-|-|-|-|-|-|}
%			&\cellcolor{light-gray} Relative change & \cellcolor{light-gray} 0.0 &\cellcolor{light-gray} 0.0 &\cellcolor{light-gray} 0.332 &\cellcolor{light-gray} 0.094 &\cellcolor{light-gray} 0.102 &\cellcolor{light-gray} \\\cline{2-8}
%			&Zero original			   & 0.0 & 0.0 &   0.105 & 0.082 & 0.123 & \\\cline{2-8}
%			&Non-zero original	    & 0.0 & 0.0 &   0.366 & 0.191 & 0.216 & \\\cline{2-8}
%			& Relative change &  0.0 & 0.0 & 0.261 & 0.109 & 0.093 & \\\hline
%		\multirow{6}{*}{Sixes data}
%			&\cellcolor{light-gray}Zero augmented 		  &\cellcolor{light-gray} 0.0  &\cellcolor{light-gray} 0.0 \cellcolor{light-gray}&\cellcolor{light-gray}   0.067 &\cellcolor{light-gray} 0.217 &\cellcolor{light-gray} 0.318 &\cellcolor{light-gray} 0.427 \\\hhline{~|-|-|-|-|-|-|-|}
%			&\cellcolor{light-gray}Non-zero augmented &\cellcolor{light-gray} 0.0 &\cellcolor{light-gray} 0.0	 &\cellcolor{light-gray} 0.667 &\cellcolor{light-gray} 0.309 &\cellcolor{light-gray} 0.398 &\cellcolor{light-gray} 0.427 \\\hhline{~|-|-|-|-|-|-|-|}
%			& \cellcolor{light-gray}Relative change &\cellcolor{light-gray} 0.0 &\cellcolor{light-gray} 0.0 &\cellcolor{light-gray} 0.600 &\cellcolor{light-gray} 0.092 &\cellcolor{light-gray} 0.080 &\cellcolor{light-gray} 0.0 \\\cline{2-8}
%			&Zero original			   & 0.0 & 0.059& 0.147 & 0.239 & 0.289 & 0.439 \\\cline{2-8}
%			&Non-zero original	    & 0.0 & 0.198& 0.367 & 0.299 & 0.362 & 0.439 \\\cline{2-8}
%			&Relative change & 0.0 & 0.139 & 0.220 & 0.060 & 0.073 & 0.0 \\\hline
%	\end{tabular}%
%	}
%	\caption{Average DICE of each 'time interval' of the five series and six series test data.}
%	\label{tableDice}
%\end{table}
\section{Discussion of results}
\subsection{Model evaluation}
In \autoref{2dresnetevalres} we see the model evaluation of models trained on the different datasets. We see the training accuracies for all models being close to 100\% while the validation accuracies are lagging behind. The gap between training and validation accuracy, and the training accuracy being this high, would indicate that the model is overfitting the training data, which is likely also the reason a lot of models overly predicts either healthy or inflamed. It would thus have been sensible to add more regularization to the models for instance in the form of weight decay. This was also attempted, however, the combination of high dropout and weight decay seemed to halt learning to a degree where both training and validation accuracy would stay around 50\%. Because of this, it was decided not to add weight decay, and due to time constraints towards the end of the thesis, this issue was never re-visited. As these models are the very foundation of the project, more should probably have been done to tune them more accurately. This could have been done by training for more epochs and accordingly lower the learning rate. In state of the art applications, models are often trained for several days with low learning rate, however, such a setting would not have been feasible for this project as models was run in Google Colab with very limiting usage restrictions. Models could, however, have been trained for a lot longer than they were with a much smaller learning rate. Different model architectures could have been attempted. This could both be a different variant of ResNet, but also a completely different architecture. Also more combinations of dropout and weight decay could have been attempted to resolve the no learning issue. Likewise, different combinations of skipping frames in the videos used for training could have been explored more.\\
Splitting all the videos into seven folds, training on six and validating on one was also attempted. This was attempted because, usually the more diverse and the larger a volume of data used for training leads to the best results when training machine learning models. However, this was extensively tested by varying epochs, learning rate, number of skipped frames, weight decay and dropout. Because the validation results would not move above 56\% the experiment was prematurely ended, and thus not reported.

An interesting tendency we see in \autoref{2dresnetevalres} and when it was attempted to split all the videos into seven folds is that, the sheer number of frames does not seem to have any relation to the model accuracies. Although it is not directly the sheer number of training examples which is the reason machine learning models in general perform better on larger datasets, but the increased likelihood of having varied data, we would still have expected to see higher accuracies when training on more data and across several patients. To some extent we do see more data increases accuracy, as the two models trained on the same datasets but with varying amounts of data, both do increase in accuracy as their training set increases in size. However, we also see model \textit{idx\_2\_3\_4\_5\_6\_skip\_50} having higher accuracy than model \textit{idx\_4\_14\_18\_20\_32\_skip\_5} although the latter model's training set is eight times larger. This might simply boil down to the validation set being more similar to the data model \textit{idx\_2\_3\_4\_5\_6\_skip\_50} is trained on, but during the visual inspection on other videos from other patients, model \textit{idx\_2\_3\_4\_5\_6\_skip\_20} (trained on similar data) still performed better than model \textit{idx\_4\_14\_18\_20\_32\_skip\_5}. This also goes to show training on more patients do not necessarily increase model accuracy either.\\
It is hard to conclude why more data and training on more patients seemingly have no or limited effect on the model accuracy. One reason could be the features found vary too much or exists in both classes, contradicting themselves, and instead of making the model more robust, confuses it. Another reason could be the degree of inflammation varies too much, making it harder to find common features, and maybe it would have been beneficial to construct datasets with the type of treatment prescribed in mind. A last reason could be the amount of noisy images sampled from the videos vary, and the datasets trained across several patients might coincidentally have more noisy frames in them, although this has not been investigated.\\
Exactly why model \textit{idx\_2\_3\_4\_5\_6\_skip\_20} performs the best is also hard to conclude, but it is likely correlated to the above.

Looking into the visual results in \autoref{firstfive} and \autoref{lastthree} we see some quite volatile predictions from most models. This is likely due to the nature of the videos, where many frames all throughout the videos are all black or red, having glare effects or having a significant colour change when the camera comes too close to the colon to be able to focus. This is not an issue for a doctor to see past, however, when the model is trained on these noisy frames which belongs to both the healthy and inflamed classes, it is naturally a hard decision for the model to classify them correctly. The ResNet model also has no notion of what happens before or after the frame it processes, which gives the model no context to work out if an all red frame came before or after other inflamed frames. Another issue stems from the motion of the camera as some frames are blurred, which might make it hard to detect the features needed to determine if the frame contains signs of inflammation. In \autoref{exampleFrames} (a) and (d) we also saw an example of two very similar frames, being of opposite classes, and one could raise the question if (a) is inflamed or the lighting simply makes it look like it. This means although it might appear the models are indecisive or having issues predicting parts of the videos due to the volatile predictions, I argue it is to some extent impossible to avoid, and should be expected. Examples of noisy frames can be seen in \autoref{noisyFrames}.

\begin{figure}
	\centering
	\begin{subfigure}{0.4\linewidth}
		\centering
		\includegraphics[width=\linewidth]{Materials/Discussion/idx_1_frame_550}
		\caption{Example of motion blur and glare (inflamed).}
	\end{subfigure}
	\hspace{0.5cm}
	\begin{subfigure}{0.4\linewidth}
		\centering
		\includegraphics[width=\linewidth]{Materials/Discussion/idx_1_frame_1410}
		\caption{Example of all red frame (healthy).}
	\end{subfigure}
	\\
	\begin{subfigure}{0.4\linewidth}
		\centering
		\includegraphics[width=\linewidth]{Materials/Discussion/idx_4_frame_750}
		\caption{Example of blurred frame (inflamed).\newline}
	\end{subfigure}
	\hspace{0.5cm}
	\begin{subfigure}{0.4\linewidth}
		\centering
		\includegraphics[width=\linewidth]{Materials/Discussion/idx_4_frame_840}
		\caption{Example of a large part of a frame changing colour to yellow (inflamed).}
	\end{subfigure}
	\caption{Examples of noisy frames in the endoscopy videos.}
	\label{noisyFrames}
\end{figure}

One could remove all of these frames from the training data, making it a lot easier for the model to find the needed features in the images to detect inflammation, however, when it is then presented with a video 'from the real world', it would likely not be able to generalize due to unseen noise.

To evaluate the performance of model \textit{Idx\_2\_3\_4\_5\_6\_skip\_20}, we take a look at its $65\%$ validation accuracy and at its predictions in \autoref{idx23}, \autoref{idx24} and \autoref{idx28}. Its accuracy is by itself not impressive, and especially not when taking into consideration about $72\%$ of the validation video are inflamed frames. Looking at its predictions, the model overly predicts inflammation, which leads to a conservative model which has a hard time predicting correctly on videos which are balanced or biased towards healthy frames. On the other hand, the model seems to perform decent on videos biased towards inflamed frames, where it most often finds some correct healthy frames. Although we should not expect perfect segmentation due to the noisy nature of the videos, we conclude none of the models are performing optimally, and model \textit{Idx\_2\_3\_4\_5\_6\_skip\_20} to a small degree produces usable results, however, they are not of a quality where they can contribute with any meaningful information to a doctor.

\subsection{Separation point predictions}
A general tendency for the Random Walker post-processing was to predict the separation point at the location of the last uncertainty in predictions. That is, the rightmost part of the segmentation where there would be either volatile predictions or elevated probability of predicting healthy. Because the Random Walker has a hard time crossing points or frames where the model predictions are dissimilar, this is somewhat expected, but due to the random noise in the videos previously discussed, this makes this approach conservative and not very robust. This was also verified as on average, the Random Walker approach would predict a separation point $51.85\%$ of the total number of frames in a video away from the true separation point. For the results produced by model \textit{Idx\_2\_3\_4\_5\_6\_skip\_20} this approach is thus not usable to predict the separation points.\\
Had we had more accurate predictions, the results would likely have been better, however, due to the noisy nature of the videos, and how the Random Walker would likely get 'stuck' on the noise, it is still questionable if this approach can produce meaningful results.

The scoring heuristic works by finding dissimilarities between the prediction probabilities of two consecutive frames. If the there is a high probability both frames belong to the same class, the score is high, whereas, if there is a high probability they belong to different classes, the score is low. Thus, this approach should be good at finding the transitions between frames where the model is insecure or predicts on noisy data. This is also generally what we see, and often one of the candidate points are near the true separation point, however, it is rare this candidate point holds the actual lowest score. As it on average would predict a separation point $51.87\%$ of the total number of frames in a video away from the true separation point, its 'believed separation point' is not usable to predict the separation points.\\
Providing candidate points could be useful for a doctor to do a quick evaluation based on the predicted segmentation and the candidate points. This could save the doctor time by not having to look the whole video through. But, as it is hard, if not impossible, to tell if any of the candidate points actually resembles the true separation point, only having the predicted segmentation available, the fact one of the candidate points often (but not always) is accurate is not of much help in practice.

\subsection{Treatment predictions}
While most treatment annotations are clear, some videos have been annotated with treatment 'x or y'. These cases are indices $7$ and $8$ which have been annotated as class $2$ or $3$, index 14 annotated class $2$ or $4$, index 28 annotated class $0$ or $1$ and indices 34 and 35 annotated class $1$ or $2$. In all these cases, the largest class number has been chosen as the true class, however, this might add or subtract a few percentages of accuracy depending on how the true classes are chosen.

In \autoref{treatmentTable} we saw the 1D ResNet model could achieve an accuracy of $40\%$, which is good considering random predictions would have yielded $20\%$ accuracy. However, if these predictions were to be used by a doctor, as for instance a second opinion, $40\%$ is rather low and would be too unreliable. We also see the model only predicting classes $0$ and $2$, which would indicate the model does not generalize too well.\\
Looking at \autoref{rocpreds} we see class $0$ and class $1$ having an area under the curve above $0.5$ which indicates the models have some measure of class separability whereas for classes $2$ and $3$ we see the scores being very close to $0.5$. Looking at class $4$, we have a score of $0.33$ which would indicate it is quite hard to predict correctly. When we look at class $3$ and $4$, we need to remember only 2 and 1 examples exists in the data, and thus the model will naturally have a hard time predicting these classes. If we look at the micro average, we see a score of $0.68$, which would indicate we can expect a model to have some measure of class separability when trained on this data, which we already have seen was the case. However, looking at the macro average which takes the class imbalance into consideration, we should not expect any meaningful results. This might, however, be skewed as class $3$ and $4$ have this few examples.

In \autoref{treatmentTableWithTrue} we added the true segmentations during training, and the accuracy increased to $48\%$. However, the model still only predicts class $0$ and $2$, indicating it overfits the training data. Looking at \autoref{rocpredsandtrue} we see the scores for class $0$ and $1$ almost swap, and the rest of the classes and the macro average staying very similar to what we had without adding the true segmentations. The micro average falls slightly, but still is at a level where some notion of class separability would be expected. The micro average falling, and yet the multi class classifier achieving a higher accuracy could be caused by the model predictions being closer to each other, i.e. the model is not as confident in its choice of class, but still ends up choosing the correct class.

Overall it would seem more tuning could have been done to achieve better results. Although nearing $50\%$ accuracy on a $5$ class classification problem seems decent, the ROC curves are not very confident, and even the highest scores being rather mediocre. Again training for more epochs with an accordingly lower learning rate could be a possible improvement. The indications of overfitting in both results would also point towards more regularization would be needed. Adding dropout to force the models to learn a wider array of features in the data could be one choice. As we are seeing the model is capable of solving the problem to some extent, a change of model architecture is most likely not needed to raise the performance.

\subsection{U-Net for video segmentation}
Opposite to ResNet, that would make its predictions on each individual image without considering the images before and after it, the U-Net models were trained on the segmentations, and because U-Net works on several levels of abstractions it would be expected to capture the 'structure' of the segmentations, i.e. how the true segmentations are blocks of first inflamed frames and then healthy frames. This structure is to a large degree missing from the initial 2D ResNet predictions, and thus the hope with the results of using the U-Net, was to bring this structure into the predictions.\\
Looking at \autoref{unetres1} and \autoref{unetres2} we see the U-Net overly predict healthy frames, which probably is a reaction to the ResNet overly predicting inflammation. The structure we were seeking, is also only brought back to a limited extent. Because the U-Net predictions overrule the correctly predicted frames from the ResNet, it seems the problem with the original predictions have just shifted from being overly many inflamed predictions to overly many healthy predictions. 

\subsection{U-Net predictions for treatment prediction}
Although the predictions of the U-Net models did not seem to make the results much more usable than the original 2D ResNet results, we see a big difference in how the treatments are predicted. In \autoref{unetPredsTreatmentTable} we now see predictions of class 0, 1 and 2. Because of the low number of examples of class 3 and 4 treatments, it is not surprising we do not see these classes predicted. We now have an accuracy of $52\%$ which, taking into consideration we have 5 classes, is decent. However, it is still not high enough to, for instance, act as a second opinion for a doctor. Looking at \autoref{rocunetpreds} we now see a drastic drop in the area under the ROC curves. We now only have class 4 not being near a score of $0.5$. Given how the 'one-vs-rest' classifiers struggle, the high accuracy could be caused by the multi class classifier not being very confident and having almost equal probabilities for each class, but the correct class coming out on top for the most part.

In \autoref{unetPredsTrueTreatmentTable} the true segmentations were added to the training. We still see the model predicting class 0, 1 and 2, however, we now see the accuracy drop to $20\%$, indicating random guesses were made. Looking at \autoref{rocunetpredstrue} we however see class 0 having a score of $0.66$, suggesting these predictions were not entirely random. The rest of the classes do struggle however. It is likely adding the true segmentations simply confused the model during training, as the features found in the predicted segmentations did not match the features found for the true segmentations. Also, the drop in accuracy could stem from the multi class classifier not being confident in its predictions, giving almost equal probabilities to each class, but this time the correct classes would not come out on top.


















\section{Conclusion}
In conclusion we have seen our model introduces an exponentially increasing error in increased area when we apply force to our beam. On the other hand we can also conclude the 'bend-ability' of our materials in relation to each other are as expected and thus we have verified this part of our model works correctly. We have also looked at the total displacement per grid node and area of the beam when changing the grid resolution. Here we conclude that it is hard to determine how the beam would realistically bend, but we see an increase in both total displacement per node and in area when we increase the grid resolution. The increasing area contradicts our expectation of the model accuracy increasing when increasing the grid resolution. The similarity in the graphs in \autoref{measures} could also lead one to conclude that the error in total displacement per node is proportional to the error in increased area.
\newpage
\begin{thebibliography}{9}
	\bibitem{colitis}
	Healthline, \url{https://www.healthline.com/health/colitis#types-and-causes}, visited May 22 2022.
	
	\bibitem{lecun}
	LeCun et. al., 'Deep Learning', Nature Vol.521, May 2015.
	
	\bibitem{CNN}
	K. O'Shea and R. Nash, 'An Introduction to Convolutional Neural Networks', arXiv, published: 2015.
	
	\bibitem{unet}
	Olaf Ronneberger et. al., ’U-Net: Convolutional Networksfor Biomedical Image Segmentation’, Computer Science Department and BIOSS Centre for Biological Signalling Studies, University of Freiburg, Germany, Published: 2015.
	
	\bibitem{resnetPaper}
	Kaiming He et. al., 'Deep Residual Learning for Image Recognition', Microsoft Research, December 2015.
	
	\bibitem{resnet}
	Pytorch Team, RESNET, \url{https://pytorch.org/hub/pytorch_vision_resnet/}, visited 3rd May 2022.
	
	\bibitem{resnetModel}
	Researchgate, 'A Deep Learning Approach for Automated Diagnosis and Multi-Class Classification of Alzheimer’s Disease Stages Using Resting-State fMRI and Residual Neural Networks', \url{https://www.researchgate.net/figure/Original-ResNet-18-Architecture_fig1_336642248}, visited May 9th 2022.
	
	\bibitem{crossentropy}
	Brownlee, Jason, 'A Gentle Introduction to Cross-Entropy for Machine Learning', \url{https://machinelearningmastery.com/cross-entropy-for-machine-learning/}, visited May 9th 2022.
	
	\bibitem{gradientdescent}
	Wikipedia, 'Gradient descent', \url{https://en.wikipedia.org/wiki/Gradient_descent}, visited May 13th 2022.
	
	\bibitem{adam}
	D. Kingma and J. Ba, 'Adam: A method for stochastic optimization', published as a conference paper at ICLR , 2015
	
	\bibitem{GD2DIllustration}
	A. Yadav, 'All about Gradient Descent and its variants', \url{https://medium.com/analytics-vidhya/all-about-gradient-descent-and-its-variants-d095be1a833b}, visited: May 13th 2022
	
	\bibitem{GD1DIllustration}
	K. Voogd and S. Dahrs, 'Reproducibility project: Learning to learn by gradient descent by gradient descent', \url{https://tudelftgroup7.medium.com/reproducibility-project-learning-to-learn-by-gradient-descent-by-gradient-descent-9fe43c3ef948}, visited: May 13th 2022
	
	\bibitem{dropout}
	Srivastava, et. al., 'Dropout: A Simple Way to Prevent Neural Networks from Overfitting', Journal of Machine Learning Research 15 (2014), published June 2014.
	
	\bibitem{weightdecay}
	D. Vasani, 'This thing called Weight Decay', \url{https://towardsdatascience.com/this-thing-called-weight-decay-a7cd4bcfccab}, visited May 22nd 2022.
	
	\bibitem{normalization}
	LeCun et. al., 'Neural Networks: Tricks of the Trade', Springer 2nd edition, pages 16-17.
	
	\bibitem{errorsurface}
	M. Stewart, 'Neural Network Optimization', \url{https://towardsdatascience.com/neural-network-optimization-7ca72d4db3e0}, visited: May 29th 2022.
	
	\bibitem{batchnormalization}
	S. Ioffe and C. Szegedy, 'Batch Normalization: Accelerating Deep Network Training by Reducing Internal Covariate Shift', ICML, 2015.
	
	\bibitem{ROC}
	S. Narkhede, 'Understanding AUC-ROC Curve', \url{https://towardsdatascience.com/understanding-auc-roc-curve-68b2303cc9c5}, visited May 22 2022.
	
	
\end{thebibliography}

\end{document}