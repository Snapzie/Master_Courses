\section{Exercise 1}
\subsection*{1.1}
%When using homogenous coordinates the formulas of the transformation becomes simpler, more intuitive and often more symmetric, and can be written in small matrices. Also the notation becomes easier to read. The only slight downside is the conversion between Cartesian and homogenous coordinates, which however is very easy to perform.
When using homogenous coordinates we can dot the transformation matrices together to form a single transformation rather than having to apply several transformation in turn. The only slight downside is the conversion between Cartesian and homogenous coordinates, which however is very easy to perform.

\subsection*{1.2}
A rotation has one degree of freedom namely $\theta$, scaling has 2 degrees of freedom, the x scale and the y scale, and translation has 2 degrees of freedom, the x translation and the y translation. In total we thus have 5 degrees of freedom. The points needed to determine the translation is 1, as we can simply move the the point that is to be aligned. The rotation needs one more point to be determined as we can rotate around the point we translated until the second point is in place. And scaling takes another point to determine as the scaling could be negative and thus be mirrored. In total we thus need $N = 3$  to determine the alignment.