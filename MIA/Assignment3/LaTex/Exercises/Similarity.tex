\section{Similarity}
The image registration problem can be written as an optimization problem where we would like to maximize / minimize some measure of similarity. To name a few similarity measures we have \textit{sum of squared differences} (SSD), \textit{generalized p-norm}, \textit{normalized cross correlation} (NCC), \textit{mutual information} (MI) and \textit{normalized mutual information} (NMI). The first two needs to be minimized as they describe the distance between two compared pixels, whereas the rest needs to be maximized. All similarity measures play the same role in the optimization problem, to measure the similarity between the two images. Which similarity measure to use depends on the task at hand. Because SSD, P-norm and NCC uses pixel intensities to compute the similarities they are not intensity invariant, and so, they can not measure similarity between images of different modalities. On the other hand, MI and NMI uses distributions to compute the similarities, and so, as long as the distribution of pixel intensities stays the same, it does not matter which intensity they have, and so MI and NMI are intensity invariant and can be used to compare images of different modalities. This means the similarity measure we choose plays a role in which types of images we can compare. On the flip side, SSD is a very simple quadratic function, and we would expect this to be much easier and faster to optimize than the more complex NMI, and so the similarity measure we choose also has a role in the speed of our registration.