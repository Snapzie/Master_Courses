\section{Conclusion}
In conclusion we have talked about the registration problem being an optimization problem where the goal is to align two images' main features, we have discussed that the similarity measures play a role in which images we can compare and how fast we can compare them. We also conclude regularization can be used to penalize unwanted behaviour in our optimizations. We have also discussed the differences between rigid, affine and non-rigid transformations, where we should use each, and that the more parameters we optimize for, the greater deformations we can perform, but the harder it becomes to optimize. Lastly we have seen two examples of non-rigid registrations and illustrated the importance of choosing the correct similarity measure for the task at hand.