\section{Medical image segmentation}
Segmentation is widely used in medical image analysis where knowledge of an object's shape, size or volume is needed. It could be there is a suspicion that a patient's heart is enlarged, then finding the shape of the heart from an x-ray image would allow for easy measurement of its size. However, this would require someone to draw the shape of the heart from that x-ray scan. With the development of segmentation algorithms this can now be automated. Medical image segmentation can also be used for imaged guided procedures, where a surgeon would have to operate with great precision, for example when inserting pronator screws. Medical image segmentation is also, as touched upon before, more broadly used for computer aided diagnosis where perhaps the shape of a patient's organs is measured and used for a diagnosis. Other examples would be to apply thresholding on a CT image to extract the bones, to apply Region growing in mammograms in order to extract a potential lesion from its background, and to separate different tissues from each other based on EM results \cite{otherExamples}.

%But how is medical image segmentation achieved? There are numerous methods all with their own pros and cons. Some of them will be discussed in the following.\\
%\textbf{Histogram thresholding} is a strong tool to identify objects based on their relative intensity to other objects in an image. In an image with high contrast, the background might be black while the object we would like to segment is completely white, then the histogram of this image would show two strong peaks in the lower and upper intensities. Knowing the object we want to segment is white, we would simply threshold the image and we would have a good segmentation. However, in reality object are rarely this easy to segment due to other objects in the image having the same intensity or due to noise in the image. However, this is usually a good starting point for a segmentation.\\
%\textbf{Region growing} is an algorithm which given some initial starting point grows to neighbouring pixels in a 4- or 8-neighbourhood based on the distance in intensity to that neighbour. If the distance is smaller than some threshold, then the neighbour is 'adopted' and the segmentation 'grows'. This continues until all neighbours have been explored. This method often leaves holes in the segmentation or produces leaks due to noise. Other similar methods of segmentation to region growing are Otsu, watershed and k-means.\\
%\textbf{Connected component decomposition} is used to separate disconnected object from each other. This could for instance be done after thresholding where several objects with similar intensities might have been found. In a lung segmentation the lungs often have similar intensities to the background due to air, and thus after a histogram thresholding one might have several objects. Connected component decomposition then separates these object into their own 'component' and one can then for instance choose to keep the two largest components, which would be expected to be the lungs.\\
%\textbf{Dilation and erosion} is used to close an object and to remove leaks respectively. Dilation adds a layer of boundary to an object, which means if it contains holes they get closed. Erosion removes a layer of boundary, which means if the object have leaks, they get disconnected or even completely removed from the main object. The two operations are often used in conjunction to close the object (dilation) and then restore it again (erosion) or to remove leaks (erosion) and then restore it again (dilation).\\
%\textbf{Graph cut} is used to find object boundaries. Here the image is made to a graph representation and the more similar neighbouring pixels are, the more flow we allow between them. A max flow problem is then solved and a min cut is made to find the object boundary. In this approach several pixels need to be mark to initialize the algorithm.\\
%\textbf{Random walker} is an algorithm which given seeds denoting different object classes has a 'walker' starting at a pixel location walking around the image with probabilities of where he is going proportional to the similarity between the pixel he is in and the neighbouring pixels. When he meets a seed, the class of the seed is recorded and after \textit{x} trials, the walker will begin in the next pixel of the image. Each pixel then gets the class its walkers reached most times out of the \textit{x} trials. This gives a segmentation of the whole image into any number of classes, but requires seeds are placed in the image beforehand.\\
%\textbf{Active shape models} 
