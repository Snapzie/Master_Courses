\subsection{Error measures}
We can now discuss the definition of each error measure.\\
\textit{Mean localization error} is defined in accordance to \cite{sp} where we only use true positive detections and find the average distance between their nearest neighbours. This measure tells us about how close the key points are to each other in the two images, that is, if the image descriptor finds the same points in the two images. The error ranges between 0 and $\epsilon$, where $\epsilon$ is the maximum distance between the two key points there can be for them to be considered true positives. As it is unclear which epsilon is used in \cite{sp}, I have chosen to report the mean localization error for a range of epsilon values. To avoid rewarding image descriptors which find 0 features, the maximum error ($\epsilon$) is used for a run where 0 features are found.\\
\textit{Nearest neighbour mean average precision} is defined in accordance to \cite{sp}, and takes all descriptors of an image pair and finds their nearest neighbour. Then creates a precision / recall curve and measures the area under the curve for this image pair. Then the average area under the curve is estimated across all image pairs. This measure estimates how well the image descriptor detects true positives without introducing false positives. The measure ranges between 0 and 1. As we are comparing the same scene, the descriptions made should describe the same image patches, and thus we do not perform any perspective warping for this measure. As it is not clear from \cite{sp} which range of $\epsilon$ values are used, I have chosen to use 100 evenly spaced $\epsilon$ values between 100 - 800 as the distance between all found descriptors in the manually inspected images seem to be matched in this range. To make this measure computational feasible, only the first 1000 descriptors found are used. These are likely in the same part of the image, but sampling 1000 descriptors across the image might result in a lot of descriptors which are not measured at the same key points.\\
\textit{Homography estimation} is defined in accordance to \cite{sp} and takes the four corners of \textit{image 1} and first measures the average distance between each corresponding corner transformed by the true homography matrix and the estimated homography matrix found by taking nearest neighbour matches found by the image descriptor. If the average distance is less than some $\epsilon$ then it is reported as a correct homography estimation, otherwise it is reported as a wrong estimation. The average number of correct estimations over all image pairs is then reported as the \textit{homography estimation} measure. This measure evaluates the image descriptor as a whole by taking points matched in the two images to estimate a homography to warp one image into the perspective of the other. The result ranges between 0 and 1. As SIFT returns \textit{a lot} of features, a Lowe's ratio at $0.7$ is used to filter the number of features used to estimate the homography.